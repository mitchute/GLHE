%% This is file `elsarticle-template-1-num.tex',
%%
%% Copyright 2009 Elsevier Ltd
%%
%% This file is part of the 'Elsarticle Bundle'.
%% ---------------------------------------------
%%
%% It may be distributed under the conditions of the LaTeX Project Public
%% License, either version 1.2 of this license or (at your option) any
%% later version.  The latest version of this license is in
%%    http://www.latex-project.org/lppl.txt
%% and version 1.2 or later is part of all distributions of LaTeX
%% version 1999/12/01 or later.
%%
%% Template article for Elsevier's document class `elsarticle'
%% with numbered style bibliographic references
%%
%% $Id: elsarticle-template-1-num.tex 149 2009-10-08 05:01:15Z rishi $
%% $URL: http://lenova.river-valley.com/svn/elsbst/trunk/elsarticle-template-1-num.tex $
%%
\documentclass[review,12pt]{elsarticle}

%% Use the option review to obtain double line spacing
%% \documentclass[preprint,review,12pt]{elsarticle}

%% Use the options 1p,twocolumn; 3p; 3p,twocolumn; 5p; or 5p,twocolumn
%% for a journal layout:
%% \documentclass[final,1p,times]{elsarticle}
%% \documentclass[final,1p,times,twocolumn]{elsarticle}
%% \documentclass[final,3p,times]{elsarticle}
%% \documentclass[final,3p,times,twocolumn]{elsarticle}
%% \documentclass[final,5p,times]{elsarticle}
%% \documentclass[final,5p,times,twocolumn]{elsarticle}

%% The graphicx package provides the includegraphics command.
\usepackage{graphicx}
%% The amssymb package provides various useful mathematical symbols
\usepackage{amssymb}
\usepackage{amsmath}
%% The amsthm package provides extended theorem environments
%% \usepackage{amsthm}

\newcommand{\D}[2]{\frac{\partial #1}{\partial #2}}

\usepackage{siunitx}
\sisetup{per-mode = symbol}

\usepackage[usenames,dvipsnames,table,xcdraw]{xcolor}

\usepackage{multirow}
\usepackage{url}

%% The lineno packages adds line numbers. Start line numbering with
%% \begin{linenumbers}, end it with \end{linenumbers}. Or switch it on
%% for the whole article with \linenumbers after \end{frontmatter}.
\usepackage{lineno}

%% natbib.sty is loaded by default. However, natbib options can be
%% provided with \biboptions{...} command. Following options are
%% valid:

%%   round  -  round parentheses are used (default)
%%   square -  square brackets are used   [option]
%%   curly  -  curly braces are used      {option}
%%   angle  -  angle brackets are used    <option>
%%   semicolon  -  multiple citations separated by semi-colon
%%   colon  - same as semicolon, an earlier confusion
%%   comma  -  separated by comma
%%   numbers-  selects numerical citations
%%   super  -  numerical citations as superscripts
%%   sort   -  sorts multiple citations according to order in ref. list
%%   sort&compress   -  like sort, but also compresses numerical citations
%%   compress - compresses without sorting
%%
%% \biboptions{comma,round}

% \biboptions{}

\graphicspath{{Figures/}}

\journal{ASHRAE Transactions}

\begin{document}

\begin{frontmatter}

%% Title, authors and addresses

\title{Characterization, Testing and Optimization of Load Aggregation Methods for Ground Heat Exchanger Response-Factor Models}

%% use the tnoteref command within \title for footnotes;
%% use the tnotetext command for the associated footnote;
%% use the fnref command within \author or \address for footnotes;
%% use the fntext command for the associated footnote;
%% use the corref command within \author for corresponding author footnotes;
%% use the cortext command for the associated footnote;
%% use the ead command for the email address,
%% and the form \ead[url] for the home page:
%%
%% \title{Title\tnoteref{label1}}
%% \tnotetext[label1]{}
%% \author{Name\corref{cor1}\fnref{label2}}
%% \ead{email address}
%% \ead[url]{home page}
%% \fntext[label2]{}
%% \cortext[cor1]{}
%% \address{Address\fnref{label3}}
%% \fntext[label3]{}


%% use optional labels to link authors explicitly to addresses:
%% \author[label1,label2]{<author name>}
%% \address[label1]{<address>}
%% \address[label2]{<address>}

\author[label1]{Matt Mitchell\corref{cor1}}
\author[label1]{Jeffrey Spitler}
\author[label2]{Edwin Lee}

\address[label1]{Oklahoma State University, Stillwater OK.}
\address[label2]{National Renewable Energy Laboratory, Golden CO.}

\cortext[cor1]{Corresponding author: matt.s.mitchell@okstate.edu}

\begin{abstract}
%% Text of abstract
Ground heat exchangers are often simulated using response factor-type models. To compute the GHE temperature response, these models rely on the GHE load history and a series of temperature response factors. As the simulation progresses, the number of GHE history terms which need to be included in the temperature response computations continues to grow. The result of this is that the number of computations required grows with the square of the number of simulation time-steps. For extended, multi-year simulations the computational time required can be excessive which limits the response factor model's usefulness.

In order to reduce the number of computations required, a number of load aggregation methods have been developed. These load aggregation methods allow loads which occurred at times sufficiently far from the present simulation time to be aggregated and combined together. Therefore, the net effect of those loads is preserved, but the individual contribution of each load is not. This reduces the number of computations required and results in a faster simulation runtime.

A number of different load aggregation methods have been developed, but they have not been compared directly or been optimized for performance. In this work, the load aggregation methods are implemented and compared. A parametric study is then performed and the optimum method and operation parameters are given.

\end{abstract}

\begin{keyword}
Ground heat exchanger, simulation, load aggregation, g-function
\end{keyword}

\end{frontmatter}

%%
%% Start line numbering here if you want
%%
\linenumbers

%% main text
\section{Introduction}
Ground source heat pump (GSHP) systems are used to provide building space heating and cooling and hot water generation. In these systems, a heat pump is coupled to the soil using a ground heat exchanger (GHE), which is the heat sink/source for the system. This system then exchanges energy between the heat pump and the soil via the GHE to provide for the heating and cooling loads. Below a certain depth (which varies with geography and location) the soil temperature remains constant during the year. This temperature is typically lower than the local ambient air temperature during periods of peak cooling demand, and higher than the ambient air temperature during periods of peek heating demand. That, coupled with the fact that electricity can be generated cheaply using renewable energy sources allows GSHP systems to generally provide heating and cooling more efficiently and more economically than by using other more conventional heating and cooling technologies.

GSHP systems can be an effective and economical method for providing space heating and cooling. However, due to the complex nature of the thermally interacting ground heat exchangers, they are often difficult to apply rule-of-thumb based design approaches as could more easily be done with other heating and cooling technologies. If the ground heat exchanger is not designed properly it can cause the entire system to fail to meet the load. Since the GHE represents potentially the largest single-item cost of a GSHP system, it is important for it to be designed properly.

Therefore, in order for GSHP systems to be designed properly and compared against other comparable heating and cooling systems, system designers need the ability to simulate and compare the competing systems. Due to this, a great amount of effort has been put towards developing simulation models for GSHP and GHE systems. Designers, however, not only expect that the simulations be accurate, but they also expect them to be executed quickly.

One such GHE model that is often used for performing hourly GHE simulations is the model originally developed by \cite{EskilsonClaesson1988}. This model relies on precomputed response factors and the GHE load history to compute the temperature response of the GHE. Response factor-type models are typically formulated as is shown in Equation \ref{eq:agg:MFT}.

\begin{equation}
    T_f = T_s + \sum_{i=1}^n \frac{q_{i} - q_{i-1}}{2 \cdot \pi \cdot k_s} \cdot g\left(\frac{t_n - t_{i-1}}{t_s}\right) +  q_n \cdot R_b
    \label{eq:agg:MFT}
\end{equation}

where:
\begin{align*}
    T_f & \mbox{ --- } \mbox{GHE mean-fluid temperature} && \\
    T_s & \mbox{ --- }\mbox{soil temperature} && \\
    q_i & \mbox{ --- }i^{th} \mbox{ GHE heat transfer rate per unit GHE length. i.e Btu/h-ft, \SI{}{\watt\per\meter}} && \\
    t_i & \mbox{ --- }i^{th} \mbox{ time-step} && \\
    t_s & \mbox{ --- }\mbox{ GHE time constant } \left(\frac{H^2}{9\cdot\alpha_g}\right) && \\
    R_b & \mbox{ --- }\mbox{ borehole thermal resistance} && \\
    g & \mbox{ --- }\mbox{g-function response factors} && \\
\end{align*}

At a fundamental level, conduction problems can be evaluated using partial differential equations, as is seen in the simplified heat equation in radial coordinates given in Equation \ref{eq:agg:general conduction simplified}. The heat equation is linear and inhomogeneous, and as a result Duhamel's principle can be applied which allows the equation to be broken down into a set of homogeneous equation that correspond to a given time interval \citep{Cullin2014}.

\begin{equation}
	\frac{1}{r} \D{}{r} \left(r \D{T}{r}\right) = \frac{1}{\alpha} \D{T}{t}
	\label{eq:agg:general conduction simplified}
\end{equation}

As a result, the principle of superpostion can be applied to simplify the solution. GHE response factor models take advantage of this property and apply the principle of superposition to compute the GHE temperature response by using the GHE load history. The GHE load history is treated as a series of step heat pulses which when superimposed and combined using the response factors can lead to an accurate calculation of the GHE temperature response. This reduces the solution of the fundamental conduction differential equations to the simple algebraic equation shown in Equation \ref{eq:agg:MFT}.

A practical issue that arises from this superpostion approach, however, is the fact that the number of superposition calculations grows with the square of the number of time-steps \citep{YavuzturkSpitler1999}. As a result, applying this approach directly will greatly affect simulation runtime. Attempting to simulate, for example, an hourly annual or hourly multi-year simulations may be impractical without some way to reduce the number of computations required.

In order to reduce runtime, load aggregation procedures have been developed which will reduce then number of superposition calculations, which will in turn reduce simulation runtime. Load aggregation is the process of taking discrete, individual loads that occurred over a given time interval and replacing them with a single load over the same time interval. If the individual loads occurred over uniform time-step intervals the average load rate could be computed and applied for the aggregated load. If the loads have non-uniform time-steps an energy-balance approach should be taken to ensure that energy is conserved.

A load aggregation example is described here and in Table \ref{tab:agg example}. The most current simulation time is represented by bin 1, which represents simulation time $t_n$. This bin\footnote{The containers which are used to aggregate GHE load information will generally be referred to a ``bins" in this work.} has a width of one time-step, $h$.

For this example, let us assume that at each time-step a load with a value of \SI{1}{\watt} occurs and that the simulation time-step, $h$, is \SI{1}{\second}. Naturally, this represents \SI{1}{\joule} of energy which is stored in bin 1. As time-steps occur, existing loads are displaced and shifted right farther into the GHE's load history. Further descriptions of the aggregation example are given below.

\begin{table}[htbp!]
\centering
\caption{Average load values for the aggregation example}
\label{tab:agg example}
\begin{tabular}{|c|c|c|c|c|}
\hline
Sim Time     & $t_n$    & $t_{n-1}$    & $t_{n-2}$ - $t_{n-4}$ & $t_{n-4}$ - $t_{n-6}$ \\
Bin Width    & $h$      & $h$          & $2h$                  & $2h$                  \\
Bin Number   & 1        & 2            & 3                     & 4                     \\
Time-step    &          &              &                       &                       \\ \hline
1            & 1        &              &                       &                       \\ \hline
2            & 1        & 1            &                       &                       \\ \hline
3            & 1        & 1            & 1/2                   &                       \\ \hline
4            & 1        & 1            & 1                     &                       \\ \hline
5            & 1        & 1            & 1                     & 1/2                   \\ \hline
6            & 1        & 1            & 1                     & 1                     \\ \hline
\end{tabular}
\end{table}

\begin{itemize}
    \item Time-step 1: the bin 1 is filled with a load of value \SI{1}{\joule} which occurred at $t_n$.
    \item Time-step 2: the load in bin 1 is displaced into bin 2. A new load which occurs at the current time-step, $t_n$, is placed in bin 1.
    \item Time-step 3: the load of \SI{1}{\joule} is moved from bin 2 to bin 3. Bin 3 has a width of two time-steps, $2h$, therefore the \SI{1}{\joule} of energy which came from bin 2 now represents an average load over the bin's time period of \SI{1/2}{\watt}.
    \item Time-steps 4: another \SI{1}{\joule} of energy is added and the remaining loads are displaced. Bin 3 now contains \SI{2}{\joule} of energy for an average load of \SI{1}{\watt}.
    \item Time-steps 5-6: the previous process is repeated until the all bins have an average load value of \SI{1}{\watt}, even though bins 3 and 4 each contain \SI{2}{\joule} of energy.
\end{itemize}

In the above example the loads from 6 time-steps are aggregated into 4 bins. When considering this example along with Equation \ref{eq:agg:MFT}, we can see that some computations have been eliminated. Some questions arise from this, though, which are given here.

\begin{enumerate}
    \item \textit{What is the best method for transitioning loads from one bin to another?} In the above example, the loads transitioned perfectly as is seen at time-steps 4 and 5. At time-step 4, the \SI{1}{\watt} load which occurred at $t_{n-4}$ was retained. At time-step 5, the same \SI{1}{\watt} load (which now occurred at $t_{n-5}$) is shifted into the next bin. However, in practice, this is likely not possible. The mean value of the bin is the only information which is expected to be stored in retained. As a result the information regarding ``when" and ``how much" energy should be shifted is lost. Individual load values could be retained, but the mean value of each bin would necessarily need to be re-computed at each time-step which would defeat the purpose of the load aggregation method all together.
    \item \textit{How many aggregation bins should there be?} In this example, the number of aggregation bins was 4 with a total interval of $6h$. Would there be any advantage to having more or fewer bins?
    \item \textit{How much time should each bin represent, and how should these bins be oriented relative to one another?} This example keeps 2 bins from each level and then doubles their size from $h$ to $2h$. Is there a better expansion rate for these bins, or is this the best option?
\end{enumerate}

This study aims to answer these questions by comparing and optimizing current load aggregation methods. A number of different load aggregation methods have been proposed, however, there has not been a direct comparison of all methods which includes most the recent methods. Additionally, input parameters to the existing methods have not been optimized with the goal of minimizing simulation runtime and maximizing accuracy. The work presented here describes the process used to test these aggregation algorithms.

\section{Literature Review}

Several different load aggregation procedures have been proposed. These will be outlined and described here.

\subsection{Static Method}

\cite{YavuzturkSpitler1999} were the first to develop a load aggregation procedure to reduce simulation runtime for response factor models. The method relies on using ``monthly" heat pulses to aggregate 730 hourly loads, though the method retains 192 hourly heat pulses for the most recent 192 simulation hours. In other words, once the first 922 (192 + 730) hours have passed, the 730 hours which are farthest from the current simulation time are aggregated into monthly bins. Once another 730 hours have passed the next 730 hours are aggregated into another monthly bin. This process is repeated for the duration of the simulation.

Fundamentally, this method forms the basis for what is referred to in this work as the ``static" method. The method is characterized by smaller load bins which collapse into larger bins once a sufficient number of the smaller bins have been created. As this method is applied to the \cite{YavuzturkSpitler1999} model, the smaller hourly-bins collapsed into the larger 730-hour bins after a sufficient number of smaller bins have been created.

The method could loosely be thought of as similar to the Lagrangian approach which, in the case of fluid flow characterizations tracks individual particles or packets of fluid rather than tracking a fixed control volume through which fluid flows. In our case, however, we are concerned with tracking individual loads or load bins which are formed from previously aggregated smaller bins. Once a sufficient number of the bins have been created, they collapse into a larger bins. After a bin is formed, the load values remain within the bin until the bin is aggregated together with other similar bins.

The method is referred to as ``static" because once the loads bins have been created, the load itself does not move out of the bin. I.e. the loads are ``static" relative to the bin. Smaller bins can collapse into larger bins, but the loads themselves do not move out of the bin once the bin has been created.

To illustrate this, consider the example given in Table \ref{tab:static example}. In this example, we will assume that the time-steps, $h$, have a value of \SI{1}{\second}, and that we will keep a minimum of three $h$ before aggregating two $h$-bins into a single $2h$-bin. Shaded areas represent potential bins, but the bins do not currently exist at that time-step.

\begin{table}[htb!]
\centering
\caption{Static method aggregation example}
\label{tab:static example}
\begin{tabular}{|ccccccc|}
\hline
Bin No.                                   &                            & 1                        & 2                                             & 3 & 4                                                                    & 5                                               \\
Time-step                                 &                            &                          &                                               &                                               &                                                                      &                                                 \\ \hline \hline
\multicolumn{1}{|c|}{}                    & \multicolumn{1}{c|}{Width} & \multicolumn{1}{c|}{$h$} & \multicolumn{1}{c|}{\cellcolor[HTML]{9B9B9B}} & \multicolumn{1}{c|}{\cellcolor[HTML]{9B9B9B}} & \multicolumn{1}{c|}{\cellcolor[HTML]{9B9B9B}}                        & \cellcolor[HTML]{9B9B9B}                        \\ \cline{2-7}
\multicolumn{1}{|c|}{\multirow{-2}{*}{1}} & \multicolumn{1}{c|}{Load}  & \multicolumn{1}{c|}{1}   & \multicolumn{1}{c|}{\cellcolor[HTML]{9B9B9B}} & \multicolumn{1}{c|}{\cellcolor[HTML]{9B9B9B}} & \multicolumn{1}{c|}{\cellcolor[HTML]{9B9B9B}}                        & \cellcolor[HTML]{9B9B9B}                        \\ \hline \hline
\multicolumn{1}{|c|}{}                    & \multicolumn{1}{c|}{Width} & \multicolumn{1}{c|}{$h$} & \multicolumn{1}{c|}{$h$}                      & \multicolumn{1}{c|}{\cellcolor[HTML]{9B9B9B}} & \multicolumn{1}{c|}{\cellcolor[HTML]{9B9B9B}}                        & \cellcolor[HTML]{9B9B9B}                        \\ \cline{2-7}
\multicolumn{1}{|c|}{\multirow{-2}{*}{2}} & \multicolumn{1}{c|}{Load}  & \multicolumn{1}{c|}{0}   & \multicolumn{1}{c|}{1}                        & \multicolumn{1}{c|}{\cellcolor[HTML]{9B9B9B}} & \multicolumn{1}{c|}{\cellcolor[HTML]{9B9B9B}}                        & \cellcolor[HTML]{9B9B9B}                        \\ \hline \hline
\multicolumn{1}{|c|}{}                    & \multicolumn{1}{c|}{Width} & \multicolumn{1}{c|}{$h$} & \multicolumn{1}{c|}{$h$}                      & \multicolumn{1}{c|}{$h$}                      & \multicolumn{1}{c|}{\cellcolor[HTML]{9B9B9B}{\color[HTML]{9B9B9B} }} & \cellcolor[HTML]{9B9B9B}{\color[HTML]{9B9B9B} } \\ \cline{2-7}
\multicolumn{1}{|c|}{\multirow{-2}{*}{3}} & \multicolumn{1}{c|}{Load}  & \multicolumn{1}{c|}{0}   & \multicolumn{1}{c|}{0}                        & \multicolumn{1}{c|}{1}                        & \multicolumn{1}{c|}{\cellcolor[HTML]{9B9B9B}}                        & \cellcolor[HTML]{9B9B9B}                        \\ \hline \hline
\multicolumn{1}{|c|}{}                    & \multicolumn{1}{c|}{Width} & \multicolumn{1}{c|}{$h$} & \multicolumn{1}{c|}{$h$}                      & \multicolumn{1}{c|}{$h$}                      & \multicolumn{1}{c|}{$h$}                                             & \cellcolor[HTML]{9B9B9B}{\color[HTML]{9B9B9B} } \\ \cline{2-7}
\multicolumn{1}{|c|}{\multirow{-2}{*}{4}} & \multicolumn{1}{c|}{Load}  & \multicolumn{1}{c|}{0}   & \multicolumn{1}{c|}{0}                        & \multicolumn{1}{c|}{0}                        & \multicolumn{1}{c|}{1}                                               & \cellcolor[HTML]{9B9B9B}                        \\ \hline \hline
\multicolumn{1}{|c|}{}                    & \multicolumn{1}{c|}{Width} & \multicolumn{1}{c|}{$h$} & \multicolumn{1}{c|}{$h$}                      & \multicolumn{1}{c|}{$h$}                      & \multicolumn{1}{c|}{$2h$}                                            & \cellcolor[HTML]{9B9B9B}                        \\ \cline{2-7}
\multicolumn{1}{|c|}{\multirow{-2}{*}{5}} & \multicolumn{1}{c|}{Load}  & \multicolumn{1}{c|}{0}   & \multicolumn{1}{c|}{0}                        & \multicolumn{1}{c|}{0}                        & \multicolumn{1}{c|}{1/2}                                             & \cellcolor[HTML]{9B9B9B}                        \\ \hline \hline
\multicolumn{1}{|c|}{}                    & \multicolumn{1}{c|}{Width} & \multicolumn{1}{c|}{$h$} & \multicolumn{1}{c|}{$h$}                      & \multicolumn{1}{c|}{$h$}                      & \multicolumn{1}{c|}{$h$}                                             & $2h$                                            \\ \cline{2-7}
\multicolumn{1}{|c|}{\multirow{-2}{*}{6}} & \multicolumn{1}{c|}{Load}  & \multicolumn{1}{c|}{0}   & \multicolumn{1}{c|}{0}                        & \multicolumn{1}{c|}{0}                        & \multicolumn{1}{c|}{0}                                               & 1/2                                             \\ \hline \hline
\multicolumn{1}{|c|}{}                    & \multicolumn{1}{c|}{Width} & \multicolumn{1}{c|}{$h$} & \multicolumn{1}{c|}{$h$}                      & \multicolumn{1}{c|}{$h$}                      & \multicolumn{1}{c|}{$2h$}                                            & $2h$                                            \\ \cline{2-7}
\multicolumn{1}{|c|}{\multirow{-2}{*}{7}} & \multicolumn{1}{c|}{Load}  & \multicolumn{1}{c|}{0}   & \multicolumn{1}{c|}{0}                        & \multicolumn{1}{c|}{0}                        & \multicolumn{1}{c|}{0}                                               & 1/2                                             \\ \hline
\end{tabular}
\end{table}

The steps occurring at each time-step are described below.

\begin{itemize}
    \item Time-step 1: a new load with a value of \SI{1}{\watt} occurs. Because the time-step is \SI{1}{\second}, the energy represented by this load is \SI{1}{\joule} which is placed in bin 1.

    \item Time-step 2-4: the load from bin 1 from the previous time-step is displaced to bin 2. No new loads occurs at time-step 2, so a value of 0 is placed in bin 1. This process is repeated for for the next two time-steps until bins 1-3 each contain a value of 0, and bin 4 has a value 1.

    \item Time-step 5: no new load occurs at time-step 5, so a value of 0 is place in bin 1 and the other bins are displaced so that bins 1-4 contain a value of 0 and bin 5 contains a value representing a load of \SI{1}{\watt}, or an energy value of \SI{1}{\joule}. As was stated when the example was defined above, we plan to keep three $h$ bins and then aggregate two $h$-bins into a single $2h$-bin. As a result, the loads from bins 4 and 5 collapse into a $2h$ load bin with a value of 1/2. Again, the bin represents a width of $2h$, but contains \SI{1}{\joule} of energy, so the average load value is \SI{1/2}{\watt}.

    \item Time-step 6: no new load occurs, so a value of 0 is placed in bin 1 and the other bins are displaced as before.

    \item Time-step 7: no new load occurs, so a value of 0 is placed in bin 1 and the other bins are displaced as before. Bins 4 and 5 are again aggregated into a single bin of width $2h$. These bins contained no energy, so the value the represented is \SI{0}{\watt}.

\end{itemize}

The key questions related to the parameters controlling this method and the subsequent parametric study are:

\begin{enumerate}
    \item \textit{How many aggregation bins should be retained at each level?}
    \item \textit{How much time should bins in each aggregation level represent?}
\end{enumerate}

As was seen in the above example, a minimum of three $h$-bins were kept which then collapsed into $2h$ bins. However, for an annual hourly simulation we could just as easily have kept 2, 5, or 10 $h$-bins which collapse into $3h$, $5h$, or $10h$ sized bins. Or just about any other combination of these or other bin sizes. Our objective is to minimize simulation time and maximize accuracy, but it's not currently obvious which combination of parameters will yield this behavior.

The method developed by \cite{YavuzturkSpitler1999} can be characterized by these input parameters, which are given in Table \ref{tab:yavuzturk parameters}. At aggregation level 1, a minimum of 192 hourly-bins are kept. At aggregation level 2, the \SI{1}{\hour} bins collapse into \SI{730}{\hour} bins, when possible. The authors stated that they tested the minimum hourly history periods of 24, 192, and 730 hours. However, the method is not stated to have been optimized any further. $N_{b,i}$ represents the number of bins at the $i^{th}$-level.

\begin{table}[htbp!]
\centering
\caption{\cite{YavuzturkSpitler1999} method parameters}
\label{tab:yavuzturk parameters}
\begin{tabular}{|c|c|c|}
\hline
Level & Width (hr) & Min. $N_{b,i}$ \\ \hline \hline
1     & 1          & 192       \\ \hline
2     & 730        & N/A       \\ \hline
\end{tabular}
\end{table}

\cite{Liu2005} developed another static load aggregation scheme which is stated to be an improvement on the method by \cite{YavuzturkSpitler1999}. The author referred to the scheme as a ``hierarchical" approach and used the following parameters to guide the aggregation process.

\begin{itemize}
    \item A minimum of 12 hourly-bins are kept for the 12 most recently simulated hours.

    \item Hourly loads are then aggregated into daily, 24-hour bins. These bins are referred to as ``small" bins.

    \item Five small bins are aggregated into ``medium" bins with a minimum of three small bins kept after aggregation. In other words, once eight small bins have been created, the five which are most distant from the current simulation time are aggregated into medium bins to represent a period of 120 hours.

    \item 73 medium bins are then aggregated into ``large" bins which represents 8760 simulation hours, with 40 medium bins to be kept unaggregated. In other words, once 113 medium bins have been created, the 73 which are most distant from the current simulation time are aggregated into a period of 8760 hours.

\end{itemize}

These are summarized in the simulation case descriptions in Table \ref{tab: hierarchical cases}.

\begin{table}[htbp!]
\centering
\caption{\cite{Liu2005} method simulation case descriptions}
\label{tab: hierarchical cases}
\begin{tabular}{|c|c|c|}
\hline
Level & Width (hr) & Min. $N_{b,i}$ \\ \hline \hline
1     & 1          & 12        \\ \hline
2     & 24         & 3         \\ \hline
3     & 120        & 40        \\ \hline
4     & 8760       & N/A       \\ \hline
\end{tabular}
\end{table}

The author claims that the method results in generating only 12\% of the aggregation bin when compared to the \cite{YavuzturkSpitler1999} method, and that this results in a 20\% reduction in simulation time.

\cite{BernierPinelLabibPaillot2004} also developed another static aggregation algorithm which was termed the ``multiple-load aggregation algorithm" (MLAA). The method uses a GHE model formulation which relies on a cylindrical heat source equation rather than the response factor approach outlined previously. Regardless of this, though, the load aggregation procedure which is described is applicable to the work performed here.

The authors tested a number of different variations for bin duration, but settled on the following:

\begin{itemize}
    \item A minimum of 12 hourly-bins are kept.

    \item Hourly-bins are aggregated into bins representing 48 hours, termed ``daily" bins.

    \item Daily bins are aggregated into 168 hour ``weekly" bins.

    \item Weekly bins are aggregated into 360 hour ``monthly" bins.

\end{itemize}

\begin{table}[htbp!]
\centering
\caption{\cite{BernierPinelLabibPaillot2004} original aggregation parameters}
\label{tab: MLAA parameters}
\begin{tabular}{|c|c|c|}
\hline
Level & Width (hr) & $N_{b,i}$ \\ \hline \hline
1     & 1          & 12        \\ \hline
2     & 48         & Unknown   \\ \hline
3     & 168        & Unknown   \\ \hline
4     & 360        & N/A       \\ \hline
\end{tabular}
\end{table}

Something that should be pointed out regarding this work's testing of the MLAA method is the fact that the aggregation bins specified by the authors do not aggregate together evenly. As the method is formulated in this work, the aggregation bins must be integer combinations of each other. For example, we can take several smaller bins and combine them together to form a larger bin, but we cannot take smaller bins and a fractional bin to form a larger bin. In the case of the MLAA algorithm, 48 is not an integer multiple of 168 so this discrepancy causes a minor implementation issue. From a practical stand point, it would be possible to combine smaller bins and a fractional bin together, since we would just include the fraction of the bin's energy which is proportional to the bin fraction we are including. However, this would force us to recompute all subsequent bins which is an added computational expense. The authors also do not say whether any holding periods were used besides the one mentioned for the hourly loads.

To get around these problems, simulation cases were created to bracket the intervals specified by the authors. These are specified below in Table \ref{tab:MLAA min num bins} and Table \ref{tab:MLAA bin widths}. Cases 1-4 show the minimum number of bins which will be held unaggregated for each level, and cases a-c show the duration in hours for each bin. All combinations of cases 1-4 and a-c were run in an attempt to bracket the original method.

\begin{table}[htbp!]
\centering
\caption{\cite{BernierPinelLabibPaillot2004} minimum number of bins, $N_{b,i}$}
\label{tab:MLAA min num bins}
\begin{tabular}{|ccccc|}
\hline
Level                   & 1                       & 2                      & 3                      & 4   \\
Case                    &                         &                        &                        &     \\ \hline \hline
\multicolumn{1}{|c|}{1} & \multicolumn{1}{c|}{12} & \multicolumn{1}{c|}{4} & \multicolumn{1}{c|}{3} & N/A \\ \hline
\multicolumn{1}{|c|}{2} & \multicolumn{1}{c|}{12} & \multicolumn{1}{c|}{4} & \multicolumn{1}{c|}{3} & N/A \\ \hline
\multicolumn{1}{|c|}{3} & \multicolumn{1}{c|}{12} & \multicolumn{1}{c|}{4} & \multicolumn{1}{c|}{4} & N/A \\ \hline
\multicolumn{1}{|c|}{4} & \multicolumn{1}{c|}{12} & \multicolumn{1}{c|}{1} & \multicolumn{1}{c|}{1} & N/A \\ \hline
\end{tabular}
\end{table}

\begin{table}[htbp!]
\centering
\caption{\cite{BernierPinelLabibPaillot2004} bin widths (\SI{}{\hour})}
\label{tab:MLAA bin widths}
\begin{tabular}{|ccccc|}
\hline
Level                   & 1                      & 2                       & 3                        & 4   \\
Case                    &                        &                         &                          &     \\ \hline \hline
\multicolumn{1}{|c|}{a} & \multicolumn{1}{c|}{1} & \multicolumn{1}{c|}{48} & \multicolumn{1}{c|}{144} & 432 \\ \hline
\multicolumn{1}{|c|}{b} & \multicolumn{1}{c|}{1} & \multicolumn{1}{c|}{48} & \multicolumn{1}{c|}{144} & 288 \\ \hline
\multicolumn{1}{|c|}{c} & \multicolumn{1}{c|}{1} & \multicolumn{1}{c|}{48} & \multicolumn{1}{c|}{144} & 288 \\ \hline
\end{tabular}
\end{table}

One observable disadvantage of the static method is that it is difficult to store and reuse g-function values once they have been computed. As is seen in Equation \ref{eq:agg:MFT}, the g-function is referenced from the current simulation time to the end of the bin for which you are computing. Precomputing and reusing g-functions for the first set of bins is not a problem, but issues arise for subsequent bins. This is because the duration of bins can be any integer multiple of the previous bin's duration, and each bin level can have any arbitrary number of holding periods. As a result, it becomes difficult to determine beforehand the time intervals for which g-functions will be needed. Therefore, it is easier to simply compute the g-function values as needed at the beginning of each time-step rather than attempting to precompute and store them.

An example of this can be seen by referring back to Table \ref{tab:static example}. Response factors for bins 1-3 could easily be stored and reused. However, for subsequent bins, the end time of each bin varies at each time-step. For example, at time-step 4 bin 4 ends at the end of $4h$ time-steps. Advancing to time-step 5, bin 4 now ends at time-step $5h$, thus requiring the g-function be recomputed. Advancing again at time-step 6, the bin 4 now ends again back at $4h$. This effect cascades up through the list of bins, and with additional bin durations and holding periods it becomes difficult to track. It quickly becomes more efficient to simply recompute the g-function values rather than determining which of all possible combinations of g-functions to store and retrieve.

\subsection{Dynamic Method}

The aggregation method developed by \cite{ClaessonJaved2012} is the most recent load aggregation method to have been developed. The method could be characterized as an Eulerian approach, which in a fluid flow application, tracks the flow through a fixed control volume rather than tracking individual packets or particles of fluid. In our case, we predefine the bins and allow energy to move through the bins. Hence, we term the method ``dynamic" since the loads move relative to the aggregation bins. This work has taken frame of reference to be oriented around the load values, not the aggregation bins. Thus the reasoning for labeling the \cite{YavuzturkSpitler1999}-type method ``static" and the \cite{ClaessonJaved2012}-type method ``dynamic." In the case of the static method, the bins move relative to the current simulation time, but the loads themselves are fixed within the bins, whereas the dynamic method bins are fixed relative to the current simulation time but the loads move within the bins.

The dynamic method is illustrated in Table \ref{tab:dynamic example}, which is an example taken from \cite{ClaessonJaved2012}. However, to summarize the method we can say for each bin at each time-step a fraction of the energy is shifted farther into the load history. This fraction is defined by current simulation time-step and each bin's width, or time duration. So, for example, if we have a time-step of $h$, and an individual bin has a width of $10h$, one-tenth of the load is transitioned out of the bin ($h/10h = 0.1 = $10\%). If the bin width were $h$, all of the energy in the bin would be transitioned out of the bin ($h/h = 1 = $100\%).

In Table \ref{tab:dynamic example}, the values of the GHE loads for each time-step within each bin are represented. Horizontally, the bin widths are are given as $h$ or $2h$, where $h$ represents the length of the time-step. For this example, we can assume that the time-step, $h$, is \SI{1}{\second}. As before, the most recent load is always in the first bin. See the following description:

\begin{itemize}
    \item Time-step 1: a new load with a value of \SI{1}{\watt} occurs. Because the time-step is \SI{1}{\second}, the energy represented by this load is \SI{1}{\joule} which is placed in bin 1.

    \item Time-step 2: bin 1 has a width of $h$, and given that the current time-step is $h$, all of the energy from bin 1 will move to bin 2.

    \item Time-step 3: the same which occurred in time-step 2 repeated. Bin 2 has a width of $h$ and contains \SI{1}{\joule} of energy.  Given the time-step, $h$, all of the energy from bin 2 is moved to bin 3.  However, bin 3 has a width of $2h$, so the load represented by this bin is now 1/2. Again, energy is conserved but the original pulse of \SI{1}{\joule} is now spread over a $2h$ time interval.

    \item Time-step 4-7: the process described previously is repeated.

\end{itemize}

\begin{table}[htbp!]
\centering
\caption{Dynamic aggregation example}
\label{tab:dynamic example}
\begin{tabular}{cccccccc}
Bin No.                 & 1                      & 2                      & 3                         & 4                         & 5                         & 6                         & 7                         \\
Width                   & $h$                      & $h$                      & $2h$                        & $2h$                        & $2h$                   & $2h$                        & $2h$                        \\
time-step                &                        &                        &                           &                           &                           &                           &                           \\ \cline{1-2}
\multicolumn{1}{|c|}{1} & \multicolumn{1}{c|}{1} &                        &                           &                           &                           &                           &                           \\ \cline{1-3}
\multicolumn{1}{|c|}{2} & \multicolumn{1}{c|}{0} & \multicolumn{1}{c|}{1} &                           &                           &                           &                           &                           \\ \cline{1-4}
\multicolumn{1}{|c|}{3} & \multicolumn{1}{c|}{0} & \multicolumn{1}{c|}{0} & \multicolumn{1}{c|}{1/2}  &                           &                           &                           &                           \\ \cline{1-5}
\multicolumn{1}{|c|}{4} & \multicolumn{1}{c|}{0} & \multicolumn{1}{c|}{0} & \multicolumn{1}{c|}{1/4}  & \multicolumn{1}{c|}{1/4}  &                           &                           &                           \\ \cline{1-6}
\multicolumn{1}{|c|}{5} & \multicolumn{1}{c|}{0} & \multicolumn{1}{c|}{0} & \multicolumn{1}{c|}{1/8}  & \multicolumn{1}{c|}{2/8}  & \multicolumn{1}{c|}{1/8}  &                           &                           \\ \cline{1-7}
\multicolumn{1}{|c|}{6} & \multicolumn{1}{c|}{0} & \multicolumn{1}{c|}{0} & \multicolumn{1}{c|}{1/16} & \multicolumn{1}{c|}{3/16} & \multicolumn{1}{c|}{3/16} & \multicolumn{1}{c|}{1/16} &                           \\ \hline
\multicolumn{1}{|c|}{7} & \multicolumn{1}{c|}{0} & \multicolumn{1}{c|}{0} & \multicolumn{1}{c|}{1/32} & \multicolumn{1}{c|}{4/32} & \multicolumn{1}{c|}{6/32} & \multicolumn{1}{c|}{4/32} & \multicolumn{1}{c|}{1/32} \\ \hline
\end{tabular}
\end{table}

Due to the fact that the bin widths are predetermined and are fixed relative to the current simulation time (i.e. in our above example bin 3, for example, will always represents the loads which occurred from 2-4 hours previous to the current simulation time), the g-function values can be precomputed. This eliminates the need to continuously compute these values at each time-step, as was the case with the static method.

However, another issue which arises due to this method is the fact that the method introduces diffusion and dispersion errors. The diffusion errors occur due to the magnitude of the load values diffusing with time, as is seen with the amplitude of the original value which is decreasing with time. The dispersion errors occurs due to the load values dispersing into the GHE history faster than actually occurs. To illustrate this, let us consider the previous dynamic method example.

At time-step 7, the original load value of \SI{1}{\joule} is now smeared across 5 different bins. In reality, the  true value should be 1/2 and be located within bin 5. This illustrates the diffusive nature of the method. We can also see that the load value has propagated as far at $12h$ into the load history, since we have non-zero values in bins 6-7. This illustrates the dispersive nature of the method.

In reality, we expect the hours which are most near to the current simulation time to have the largest effect on the GHE short-term temperature response. This is why the aggregation methods commonly preserve nearly all of these loads unaggregated. We do not expect hourly load variations from, say, several days before the current simulation time to have a significant effect on the current short-term temperature response. What is important, however, is that energy is conserved.

Regarding the method's implementation, the dynamic method can be characterized using two parameters; the expansion rate of the bin levels, and the number of bins at each level. The number of levels can be expanded as needed to meet the requirements of the duration of the simulation.

As implemented by \cite{ClaessonJaved2012}, the dynamic method uses an expansion rate of 2, and the number of bins at each level of 5. The authors also use 16 levels as this is more than sufficient to perform a 20 year simulation, though this could be expanded or contracted as needed to appropriately fit the simulation time. This is illustrated in Table \ref{tab:claesson example}.

The duration of the bins within each level is indicated in the level bin duration column in Table \ref{tab:claesson example}. Here, the authors determine the duration of bins within each level with the following formula:

\begin{equation}
	t_b = r_e^{i - 1}
\end{equation}

where:
\begin{align*}
    t_b & \mbox{ --- duration of each bin} && \\
    r_e & \mbox{ --- expansion rate} && \\
    i & \mbox{--- level number} && \\
\end{align*}

\begin{table}[htbp!]
\centering
\caption{Dynamic method as proposed by \cite{ClaessonJaved2012}}
\label{tab:claesson example}
\begin{tabular}{|c|c|c|c|c|c|c|}
\hline
Level & Level Bin Duration & \multicolumn{5}{c|}{Simulation Hours}      \\ \hline
1     & h                  & 1      & 2      & 3      & 4      & 5      \\ \hline
2     & 2h                 & 7      & 9      & 11     & 13     & 15     \\ \hline
3     & 4h                 & 19     & 23     & 27     & 31     & 35     \\ \hline
4     & 8h                 & 43     & 51     & 59     & 67     & 75     \\ \hline
5     & 16h                & 91     & 107    & 123    & 139    & 155    \\ \hline
6     & 32h                & 187    & 219    & 251    & 283    & 315    \\ \hline
7     & 64h                & 379    & 443    & 507    & 571    & 635    \\ \hline
8     & 128h               & 763    & 891    & 1019   & 1147   & 1275   \\ \hline
9     & 256h               & 1531   & 1787   & 2043   & 2299   & 2555   \\ \hline
10    & 512h               & 3067   & 3579   & 4091   & 4603   & 5115   \\ \hline
11    & 1024h              & 6139   & 7163   & 8187   & 9211   & 10235  \\ \hline
12    & 2048h              & 12283  & 14331  & 16379  & 18427  & 20475  \\ \hline
13    & 4096h              & 24571  & 28667  & 32763  & 36859  & 40955  \\ \hline
14    & 8192h              & 49147  & 57339  & 65531  & 73723  & 81915  \\ \hline
15    & 16384h             & 98299  & 114683 & 131067 & 147451 & 163835 \\ \hline
16    & 32768h             & 196603 & 229371 & 262139 & 294907 & 327675 \\ \hline
\end{tabular}
\end{table}

The simulation hours columns indicates the end hour in to the GHE load history for which that bin is applicable. Starting at level 1, the first bin (which has a value of 1) will account for the current simulation load. Moving right, next bin will account for the load from 1-2 hours into the GHE load history. This process is repeated until the end of the level.

At level 2, the first bin (which has a value of 7) accounts for loads from 5-7 hours into the GHE load history. The next bin accounts from hours 7-9, etc.

\cite{MARCOTTE2008651} apply a method which is similar to the dynamic method, and the implementation described by the authors is very similar to the one described by \cite{ClaessonJaved2012}. The authors, however, hold 48 hourly loads unaggregated. Then from there the, authors being aggregating into 2-hour, 4-hour, 8-hour, etc. bins. A major difference, however, is that it appears that the authors are preserving the original individual loads, then combining them together to form the aggregated loads used in the response factor temperature calculations. With the static and dynamic methods described previously, this information is lost once the aggregation has occurred, so this method is not implemented in this work.

\section{Methodology}

In order to compare the methods, a parametric study was performed by sweeping the input parameters for each method over a wide range of inputs. The input variations for the static and dynamic methods are described in sections \ref{subsec:agg:method:static} and \ref{subsec:agg:method:dynamic}, respectively.

Simulations for all variations of the static and dynamic methods, as well as all aggretation methods referenced previously were simulated for 1, 5, and 10 years using balanced and imbalanced loads. The loads were generated using EnergyPlus \citep{CRAWLEY2001319} and using an input file created by \cite{Deru2011}. The balanced load simulations have approximately the same energy being rejected to, and extracted from the ground during each simulation year. The imbalanced load profile contains loads which reject much more heat to the ground than is extracted over the course of the simulation year. The balanced loads were generated by simulating the building using the Cincinnati, Ohio USA TMY3 weather file which is provided by \cite{DOEWeatherData}. The imbalanced simulation used the Miami, Florida USA TMY3 weather file which is also provided by \cite{DOEWeatherData}. The loads data generated by EnergyPlus were for a single, hourly-annual simulation, which yielded 8760 load values. These values were repeated for multi-year simulations and the ground heat exchanger flow rate was fixed at \SI{0.1}{\kilogram\per\second} for each time-step.

The simulation input file used contained a description of a small, five-zone office building which contained air heating and cooling coils. The system modeled did not contain a simulation of a ground source heat pump. The loads used were taken from the simulation by aggregating the total air-side loads for the building and scaling them proportionally so that the peak load value was appropriate given the borehole being simulated. This resulted in peak cooling or heating loads on the ground heat exchanger of \SI{5}{\kilo\watt}. The ground heat exchanger simulated is described in \cite{BEIER201855}, which is a borehole which was installed at Oklahoma State University in 1997.

Once each individual load aggregation case had been simulated, the simulation data were compared against simulations which did not incorporate load aggregation. The root-mean squared error (Equation \ref{eq:RMSE}) for the borehole mean fluid temperature and the fraction of simulation runtime were then computed for each case.

\begin{equation}
    \mbox{RMSE} = \sqrt{\frac{\sum_{i=1}^n \left(\hat{y}_i - y_i\right)^2}{n}}
    \label{eq:RMSE}
\end{equation}

where:
\begin{align*}
    \hat{y}_i & \mbox{ --- predicted value} \\
    y_i & \mbox{ --- observed value}
\end{align*}

All of the simulations were run at the OSU High Performance Computing Center on the ``Cowboy" cluster. The cluster consists of 252 standard compute nodes, each with dual Intel Xeon E5-2620 ``Sandy Bridge" hex core 2.0 GHz CPU's, with 32 GB of 1333 MHz RAM.

The aggregation study and all data analysis were done using the Python programming language. The source code for the project is available here: \url{https://github.com/mitchute/GLHE}.

\subsection{Static Method}
\label{subsec:agg:method:static}

Besides simulating the previously described methods \citep{YavuzturkSpitler1999, BernierPinelLabibPaillot2004, Liu2005}, a parametric study was performed for the static method by varying the input parameters of the static method over a wide range. These parameters are shown in Table \ref{tab:static case durations} and Table \ref{tab:static case nums}.

Table \ref{tab:static case durations} shows cases a-v which show the duration of each bin level, in hours. Starting with case ``a", the bins increase at a $2\times$ multiplier, starting from an hourly level of 1 hour, then increasing to 2, 4, 8, etc. Case ``b" uses a $3\times$ multiplier, case ``c" uses a $4\times$ multiplier; etc. Note that there is no $6\times$ multiplier case. It was not noticed until after all simulations had been completed that this had been omitted. Regardless, the results it would have produced are expected to be bracketed by the other cases.

Table \ref{tab:static case nums} shows the minimum number of hourly bins, and the minimum number of the remaining ``other" bins which will need to be held before any aggregation can occur. Cases 1-12 show the minimum number of hourly bins, and cases $\alpha$, $\beta$, $\gamma$, and $\delta$ indicate the minimum number of the remaining bins.

All possible combinations of the parameters outlined were simulated, for a total of 1056 case combinations ($22 \times 12 \times 4$) for the static method parametric study. Note that each of these cases were simulated for 1, 5, and 10 year simulations, and that they were simulated using balanced and imbalanced loads.

\begin{table}[htbp!]
\centering
\caption{Static method parametric study simulation case durations, in hours}
\label{tab:static case durations}
\begin{tabular}{ccccccc}
\multicolumn{1}{l}{}     & \multicolumn{6}{c}{Cases}                                                                                                                                                                                                                               \\
Level                    & a                          & b                                             & c                                             & d                                             & e                                             &                            \\ \cline{1-6}
\multicolumn{1}{|c|}{1}  & \multicolumn{1}{c|}{1}     & \multicolumn{1}{c|}{1}                        & \multicolumn{1}{c|}{1}                        & \multicolumn{1}{c|}{1}                        & \multicolumn{1}{c|}{1}                        &                            \\ \cline{1-6}
\multicolumn{1}{|c|}{2}  & \multicolumn{1}{c|}{2}     & \multicolumn{1}{c|}{3}                        & \multicolumn{1}{c|}{4}                        & \multicolumn{1}{c|}{5}                        & \multicolumn{1}{c|}{7}                        &                            \\ \cline{1-6}
\multicolumn{1}{|c|}{3}  & \multicolumn{1}{c|}{4}     & \multicolumn{1}{c|}{9}                        & \multicolumn{1}{c|}{16}                       & \multicolumn{1}{c|}{25}                       & \multicolumn{1}{c|}{49}                       &                            \\ \cline{1-6}
\multicolumn{1}{|c|}{4}  & \multicolumn{1}{c|}{8}     & \multicolumn{1}{c|}{27}                       & \multicolumn{1}{c|}{64}                       & \multicolumn{1}{c|}{125}                      & \multicolumn{1}{c|}{343}                      &                            \\ \cline{1-6}
\multicolumn{1}{|c|}{5}  & \multicolumn{1}{c|}{16}    & \multicolumn{1}{c|}{81}                       & \multicolumn{1}{c|}{256}                      & \multicolumn{1}{c|}{625}                      & \multicolumn{1}{c|}{2401}                     &                            \\ \cline{1-6}
\multicolumn{1}{|c|}{6}  & \multicolumn{1}{c|}{32}    & \multicolumn{1}{c|}{243}                      & \multicolumn{1}{c|}{1024}                     & \multicolumn{1}{c|}{3125}                     & \multicolumn{1}{c|}{16807}                    &                            \\ \cline{1-6}
\multicolumn{1}{|c|}{7}  & \multicolumn{1}{c|}{64}    & \multicolumn{1}{c|}{729}                      & \multicolumn{1}{c|}{4096}                     & \multicolumn{1}{c|}{15625}                    & \multicolumn{1}{c|}{\cellcolor[HTML]{9B9B9B}} &                            \\ \cline{1-6}
\multicolumn{1}{|c|}{8}  & \multicolumn{1}{c|}{128}   & \multicolumn{1}{c|}{2187}                     & \multicolumn{1}{c|}{16384}                    & \multicolumn{1}{c|}{\cellcolor[HTML]{9B9B9B}} & \multicolumn{1}{c|}{\cellcolor[HTML]{9B9B9B}} &                            \\ \cline{1-6}
\multicolumn{1}{|c|}{9}  & \multicolumn{1}{c|}{256}   & \multicolumn{1}{c|}{6561}                     & \multicolumn{1}{c|}{\cellcolor[HTML]{9B9B9B}} & \multicolumn{1}{c|}{\cellcolor[HTML]{9B9B9B}} & \multicolumn{1}{c|}{\cellcolor[HTML]{9B9B9B}} &                            \\ \cline{1-6}
\multicolumn{1}{|c|}{10} & \multicolumn{1}{c|}{512}   & \multicolumn{1}{c|}{19683}                    & \multicolumn{1}{c|}{\cellcolor[HTML]{9B9B9B}} & \multicolumn{1}{c|}{\cellcolor[HTML]{9B9B9B}} & \multicolumn{1}{c|}{\cellcolor[HTML]{9B9B9B}} &                            \\ \cline{1-6}
\multicolumn{1}{|c|}{11} & \multicolumn{1}{c|}{1024}  & \multicolumn{1}{c|}{\cellcolor[HTML]{9B9B9B}} & \multicolumn{1}{c|}{\cellcolor[HTML]{9B9B9B}} & \multicolumn{1}{c|}{\cellcolor[HTML]{9B9B9B}} & \multicolumn{1}{c|}{\cellcolor[HTML]{9B9B9B}} &                            \\ \cline{1-6}
\multicolumn{1}{|c|}{12} & \multicolumn{1}{c|}{2048}  & \multicolumn{1}{c|}{\cellcolor[HTML]{9B9B9B}} & \multicolumn{1}{c|}{\cellcolor[HTML]{9B9B9B}} & \multicolumn{1}{c|}{\cellcolor[HTML]{9B9B9B}} & \multicolumn{1}{c|}{\cellcolor[HTML]{9B9B9B}} &                            \\ \cline{1-6}
\multicolumn{1}{|c|}{13} & \multicolumn{1}{c|}{4096}  & \multicolumn{1}{c|}{\cellcolor[HTML]{9B9B9B}} & \multicolumn{1}{c|}{\cellcolor[HTML]{9B9B9B}} & \multicolumn{1}{c|}{\cellcolor[HTML]{9B9B9B}} & \multicolumn{1}{c|}{\cellcolor[HTML]{9B9B9B}} &                            \\ \cline{1-6}
\multicolumn{1}{|c|}{14} & \multicolumn{1}{c|}{8192}  & \multicolumn{1}{c|}{\cellcolor[HTML]{9B9B9B}} & \multicolumn{1}{c|}{\cellcolor[HTML]{9B9B9B}} & \multicolumn{1}{c|}{\cellcolor[HTML]{9B9B9B}} & \multicolumn{1}{c|}{\cellcolor[HTML]{9B9B9B}} &                            \\ \cline{1-6}
\multicolumn{1}{|c|}{15} & \multicolumn{1}{c|}{16384} & \multicolumn{1}{c|}{\cellcolor[HTML]{9B9B9B}} & \multicolumn{1}{c|}{\cellcolor[HTML]{9B9B9B}} & \multicolumn{1}{c|}{\cellcolor[HTML]{9B9B9B}} & \multicolumn{1}{c|}{\cellcolor[HTML]{9B9B9B}} &                            \\ \cline{1-6}
                         &                            &                                               &                                               &                                               &                                               &                            \\
                         & f                          & g                                             & h                                             & i                                             & j                                             &                            \\ \cline{1-6}
\multicolumn{1}{|c|}{1}  & \multicolumn{1}{c|}{1}     & \multicolumn{1}{c|}{1}                        & \multicolumn{1}{c|}{1}                        & \multicolumn{1}{c|}{1}                        & \multicolumn{1}{c|}{1}                        &                            \\ \cline{1-6}
\multicolumn{1}{|c|}{2}  & \multicolumn{1}{c|}{8}     & \multicolumn{1}{c|}{9}                        & \multicolumn{1}{c|}{10}                       & \multicolumn{1}{c|}{11}                       & \multicolumn{1}{c|}{12}                       &                            \\ \cline{1-6}
\multicolumn{1}{|c|}{3}  & \multicolumn{1}{c|}{64}    & \multicolumn{1}{c|}{81}                       & \multicolumn{1}{c|}{100}                      & \multicolumn{1}{c|}{121}                      & \multicolumn{1}{c|}{144}                      &                            \\ \cline{1-6}
\multicolumn{1}{|c|}{4}  & \multicolumn{1}{c|}{512}   & \multicolumn{1}{c|}{729}                      & \multicolumn{1}{c|}{1000}                     & \multicolumn{1}{c|}{1331}                     & \multicolumn{1}{c|}{1728}                     &                            \\ \cline{1-6}
\multicolumn{1}{|c|}{5}  & \multicolumn{1}{c|}{4096}  & \multicolumn{1}{c|}{6561}                     & \multicolumn{1}{c|}{10000}                    & \multicolumn{1}{c|}{14641}                    & \multicolumn{1}{c|}{20736}                    &                            \\ \cline{1-6}
                         &                            &                                               &                                               &                                               &                                               &                            \\
                         & k                          & l                                             & m                                             & n                                             & o                                             & p                          \\ \hline
\multicolumn{1}{|c|}{1}  & \multicolumn{1}{c|}{1}     & \multicolumn{1}{c|}{1}                        & \multicolumn{1}{c|}{1}                        & \multicolumn{1}{c|}{1}                        & \multicolumn{1}{c|}{1}                        & \multicolumn{1}{c|}{1}     \\ \hline
\multicolumn{1}{|c|}{2}  & \multicolumn{1}{c|}{13}    & \multicolumn{1}{c|}{14}                       & \multicolumn{1}{c|}{15}                       & \multicolumn{1}{c|}{16}                       & \multicolumn{1}{c|}{17}                       & \multicolumn{1}{c|}{18}    \\ \hline
\multicolumn{1}{|c|}{3}  & \multicolumn{1}{c|}{169}   & \multicolumn{1}{c|}{196}                      & \multicolumn{1}{c|}{225}                      & \multicolumn{1}{c|}{256}                      & \multicolumn{1}{c|}{289}                      & \multicolumn{1}{c|}{324}   \\ \hline
\multicolumn{1}{|c|}{4}  & \multicolumn{1}{c|}{2197}  & \multicolumn{1}{c|}{2744}                     & \multicolumn{1}{c|}{3375}                     & \multicolumn{1}{c|}{4096}                     & \multicolumn{1}{c|}{4913}                     & \multicolumn{1}{c|}{5832}  \\ \hline
                         &                            &                                               &                                               &                                               &                                               &                            \\
                         & q                          & r                                             & s                                             & t                                             & u                                             & v                          \\ \hline
\multicolumn{1}{|c|}{1}  & \multicolumn{1}{c|}{1}     & \multicolumn{1}{c|}{1}                        & \multicolumn{1}{c|}{1}                        & \multicolumn{1}{c|}{1}                        & \multicolumn{1}{c|}{1}                        & \multicolumn{1}{c|}{1}     \\ \hline
\multicolumn{1}{|c|}{2}  & \multicolumn{1}{c|}{19}    & \multicolumn{1}{c|}{20}                       & \multicolumn{1}{c|}{21}                       & \multicolumn{1}{c|}{22}                       & \multicolumn{1}{c|}{23}                       & \multicolumn{1}{c|}{24}    \\ \hline
\multicolumn{1}{|c|}{3}  & \multicolumn{1}{c|}{361}   & \multicolumn{1}{c|}{400}                      & \multicolumn{1}{c|}{441}                      & \multicolumn{1}{c|}{484}                      & \multicolumn{1}{c|}{529}                      & \multicolumn{1}{c|}{576}   \\ \hline
\multicolumn{1}{|c|}{4}  & \multicolumn{1}{c|}{6859}  & \multicolumn{1}{c|}{8000}                     & \multicolumn{1}{c|}{9261}                     & \multicolumn{1}{c|}{10648}                    & \multicolumn{1}{c|}{12167}                    & \multicolumn{1}{c|}{13824} \\ \hline
\end{tabular}
\end{table}

\begin{table}[htbp!]
\centering
\caption{Static method parametric study minimum number of bins}
\label{tab:static case nums}
\begin{tabular}{|c|c|ccc}
\cline{1-2} \cline{4-5}
Case & Min. $N_{b,1}$ & \multicolumn{1}{c|}{} & \multicolumn{1}{c|}{Case}                  & \multicolumn{1}{c|}{Min. $N_{b,2-n}$} \\ \cline{1-2} \cline{4-5}
1    & 4                     & \multicolumn{1}{c|}{} & \multicolumn{1}{c|}{$\alpha$} & \multicolumn{1}{c|}{4}                    \\ \cline{1-2} \cline{4-5}
2    & 8                     & \multicolumn{1}{c|}{} & \multicolumn{1}{c|}{$\beta$}  & \multicolumn{1}{c|}{6}                    \\ \cline{1-2} \cline{4-5}
3    & 10                    & \multicolumn{1}{c|}{} & \multicolumn{1}{c|}{$\gamma$} & \multicolumn{1}{c|}{8}                    \\ \cline{1-2} \cline{4-5}
4    & 20                    & \multicolumn{1}{c|}{} & \multicolumn{1}{c|}{$\delta$} & \multicolumn{1}{c|}{10}                   \\ \cline{1-2} \cline{4-5}
5    & 30                    &                       &                                            &                                           \\ \cline{1-2}
6    & 40                    &                       &                                            &                                           \\ \cline{1-2}
7    & 50                    &                       &                                            &                                           \\ \cline{1-2}
8    & 60                    &                       &                                            &                                           \\ \cline{1-2}
9    & 70                    &                       &                                            &                                           \\ \cline{1-2}
10   & 80                    &                       &                                            &                                           \\ \cline{1-2}
11   & 90                    &                       &                                            &                                           \\ \cline{1-2}
12   & 100                   &                       &                                            &                                           \\ \cline{1-2}
\end{tabular}
\end{table}

\subsection{Dynamic Method}
\label{subsec:agg:method:dynamic}

Parameters for the dynamic method parametric study were the expansion rate, number of bins in each level, $N_{b,i}$. Expansion rate was varied from 1.25 to 3.00 at an interval of 0.25, however a value of $\frac{1+\sqrt{5}}{2} = 1.618\ldots$ (golden ratio) was also included. This yields a total of 9 expansion rates. The number of levels, $N_L$ was expanded as needed to fit the simulation runtime requirements.

The number of bins at each level for the first and last levels were varied from 1-10, and also set to the number of levels, $N_L$. Because the starting level and ending level both could have a different value, the number of bins for intermediate levels were found by setting $N_{b,i}$ linearly proportional to the number of bins in the first and last aggregation levels. For example, if the starting level has $N_{b,1} = 1$ and the ending level has $N_{b,10} = 10$, and there are 10 aggregation levels, level 5 would have 5 bins, $N_{b,5} = 5$.

To further illustrate, for the example given previously by \cite{ClaessonJaved2012} and shown in \ref{tab:claesson example}, the number of bins per level is 5, $N_{b,i} = 5$, the number of levels is 16, $N_L = 16$, and expansion rate is 2.

These combinations of input parameters resulted in a total of 1089 test combination ($9 \times 11 \times 11$) for the dynamic method parametric study. As before, each of these combinations were tested for 1, 5, and 10 simulation years, and for the balanced and imbalanced loads.

\section{Results}

\subsection{General Parametric Study Results Analysis}

The simulation results for the balanced load 1-, 5-, and 10-year simulations can be seen in Figures, \ref{fig:b1 results}, \ref{fig:b5 results}, and \ref{fig:b10 results}, respectively. Similarly, the simulation results for the imbalanced load 1-, 5-, and 10-year simulations can be seen in Figures \ref{fig:i1 results}, \ref{fig:i5 results}, and \ref{fig:i10 results}, respectively.

%\begin{figure}[htbp!]
%\centering
%\includegraphics[width=0.85\textwidth]{balanced_1.pdf}
%\caption{Balanced, 1-year parametric simulation results.}
%\label{fig:b1 results}
%\end{figure}
%
%\begin{figure}[htbp!]
%\centering
%\includegraphics[width=0.85\textwidth]{balanced_5.pdf}
%\caption{Balanced, 5-year parametric simulation results.}
%\label{fig:b5 results}
%\end{figure}
%
%\begin{figure}[htbp!]
%\centering
%\includegraphics[width=0.85\textwidth]{balanced_10.pdf}
%\caption{Balanced, 10-year parametric simulation results.}
%\label{fig:b10 results}
%\end{figure}
%
%\begin{figure}[htbp!]
%\centering
%\includegraphics[width=0.85\textwidth]{imbalanced_1.pdf}
%\caption{Imbalanced, 1-year parametric simulation results.}
%\label{fig:i1 results}
%\end{figure}
%
%\begin{figure}[htbp!]
%\centering
%\includegraphics[width=0.85\textwidth]{imbalanced_5.pdf}
%\caption{Imbalanced, 5-year parametric simulation results.}
%\label{fig:i5 results}
%\end{figure}
%
%\begin{figure}[htbp!]
%\centering
%\includegraphics[width=0.85\textwidth]{imbalanced_10.pdf}
%\caption{Imbalanced, 10-year parametric simulation results.}
%\label{fig:i10 results}
%\end{figure}
%
%In all cases, the dynamic method outperforms the other methods. The method given by \cite{YavuzturkSpitler1999} fared worst with respect to runtime fraction, which is the fraction of simulation runtime required when compared to the non-aggregated simulation case. The methods given by \cite{BernierPinelLabibPaillot2004} and \cite{Liu2005} were approximately average when when comparing runtime fraction against the static method, and had low accuracy when compared to most of the static and dynamic simulations. The method given by \cite{YavuzturkSpitler1999} also had accuracy which was nearly equivalent in accuracy to the \cite{BernierPinelLabibPaillot2004} and \cite{Liu2005} methods.
%
%\subsection{Detailed Dynamic Parametric Study Data Analysis}
%
%The method originally given by \cite{ClaessonJaved2012} fared quite well when compared to the static methods, and relatively well overall when compared to the dynamic method cases. The dynamic method showed high accuracy and low runtime fractions in all cases. There were, however, many other dynamic simulation which yielded better accuracy and better runtime savings. In an attempt to characterize the parameters which cause this behavior, the Pareto frontier values for the dynamic methods for all permutations of load and simulation time were found. These Pareto values are identified and shown, along with the respective dynamic simulation data in Figures \ref{fig:b1p results}, \ref{fig:b5p results}, and \ref{fig:b10p results} for the balanced results, and Figures \ref{fig:i1p results}, \ref{fig:i5p results}, and \ref{fig:i10p results} for the imbalanced results.
%
%\begin{figure}[htbp!]
%\centering
%\includegraphics[width=0.85\textwidth]{balanced_1_pareto.pdf}
%\caption{Balanced, 1-year dynamic method simulation results with Pareto front.}
%\label{fig:b1p results}
%\end{figure}
%
%\begin{figure}[htbp!]
%\centering
%\includegraphics[width=0.85\textwidth]{balanced_5_pareto.pdf}
%\caption{Balanced, 5-year dynamic method simulation results with Pareto front.}
%\label{fig:b5p results}
%\end{figure}
%
%\begin{figure}[htbp!]
%\centering
%\includegraphics[width=0.85\textwidth]{balanced_10_pareto.pdf}
%\caption{Balanced, 10-year dynamic method simulation results with Pareto front.}
%\label{fig:b10p results}
%\end{figure}
%
%\begin{figure}[htbp!]
%\centering
%\includegraphics[width=0.85\textwidth]{imbalanced_1_pareto.pdf}
%\caption{Imbalanced, 1-year dynamic method simulation results with Pareto front.}
%\label{fig:i1p results}
%\end{figure}
%
%\begin{figure}[htbp!]
%\centering
%\includegraphics[width=0.85\textwidth]{imbalanced_5_pareto.pdf}
%\caption{Imbalanced, 5-year dynamic method simulation results with Pareto front.}
%\label{fig:i5p results}
%\end{figure}
%
%\begin{figure}[htbp!]
%\centering
%\includegraphics[width=0.85\textwidth]{imbalanced_10_pareto.pdf}
%\caption{Imbalanced, 10-year dynamic method simulation results with Pareto front.}
%\label{fig:i10p results}
%\end{figure}
%
%The Pareto optimum is a criterion established by Vilfredo Pareto which asserts that the state of a given system is Pareto optimal, and thus efficient, if and only if there are no feasible alternative states of that system in which at least one objective is better off and no one is worse off \citep{Honderich2005}. In other words, data which meet the Pareto optimum criterion cannot be moved without making at least one individual objective worse. In our case, the Pareto points identify the simulations which are better than all other simulations in either the objective to maximize accuracy (which is to minimize RMSE error) or the objective to maximize runtime savings (which is to minimize the runtime fraction).
%
%The Pareto optimum points were identified using the Python ``pareto" library \citep{PyPareto} which implements the popular Non-dominated Sorting Genetic Algorithm, NSGA-II \citep{NSGA2}. Figure \ref{fig:hist rmse} shows the mean fluid temperature RMSE broken down by simulation year for the Pareto points; Figure \ref{fig:hist num bins first level} shows a histogram of the number of aggregation bins in the first level; and Figure \ref{fig:hist num bins last level} shows a histogram of the number of aggregation bins in the last level.
%
%\begin{figure}[htbp!]
%\centering
%\includegraphics[width=0.85\textwidth]{hist_rmse.pdf}
%\caption{Histogram of RMSE MFT for the dynamic method Pareto points.}
%\label{fig:hist rmse}
%\end{figure}
%
%\begin{figure}[htbp!]
%\centering
%\includegraphics[width=0.85\textwidth]{hist_num_first_level.pdf}
%\caption{Histogram of the number of bins in the first level for the dynamic method Pareto points.}
%\label{fig:hist num bins first level}
%\end{figure}
%
%\begin{figure}[htbp!]
%\centering
%\includegraphics[width=0.85\textwidth]{hist_num_last_level.pdf}
%\caption{Histogram of the number of bins in the last level for the dynamic method Pareto points.}
%\label{fig:hist num bins last level}
%\end{figure}
%
%The data in Figure \ref{fig:hist rmse} shows that nearly all of the Pareto data show accuracy values of less than \SI{0.1}{\celsius}, which is an acceptably high accuracy level. Figures \ref{fig:hist num bins first level} and \ref{fig:hist num bins last level} show that a significant number of these Pareto simulations have starting and ending level bin numbers of $\leq 10$. There are points with the starting and ending number of bins is $>10$, but these are expected to be less frequent due to how the parametric study input parameters were defined. As a reminder, the only points with the number of bins $>$ 10 for the first or last levels is when the they are set equal to the number of levels. From this data, we may begin to see that the number of starting and ending number of bins should likely be greater than 1, but less than 10. As of yet, any advantages of having uniform vs.~non-uniform numbers of bins per level are not determined.
%
%Figures \ref{fig:b1 exp rate}, \ref{fig:b5 exp rate}, and \ref{fig:b10 exp rate} show the balanced-load simulation result for the dynamic method data colored by the expansion rate. These are plotted for the 1-, 5-, and 10-year simulations, respectively. Figures \ref{fig:i1 exp rate}, \ref{fig:i5 exp rate}, and \ref{fig:i10 exp rate} show the same data, but for the imbalanced load cases.
%
%\begin{figure}[htbp!]
%\centering
%\includegraphics[width=0.85\textwidth]{balanced_1_exp_rate.pdf}
%\caption{Balanced, 1-year dynamic method result colored by the expansion rate.}
%\label{fig:b1 exp rate}
%\end{figure}
%
%\begin{figure}[htbp!]
%\centering
%\includegraphics[width=0.85\textwidth]{balanced_5_exp_rate.pdf}
%\caption{Balanced, 5-year dynamic method result colored by the expansion rate.}
%\label{fig:b5 exp rate}
%\end{figure}
%
%\begin{figure}[htbp!]
%\centering
%\includegraphics[width=0.85\textwidth]{balanced_10_exp_rate.pdf}
%\caption{Balanced, 10-year dynamic method result colored by the expansion rate.}
%\label{fig:b10 exp rate}
%\end{figure}
%
%\begin{figure}[htbp!]
%\centering
%\includegraphics[width=0.85\textwidth]{imbalanced_1_exp_rate.pdf}
%\caption{Imbalanced, 1-year dynamic method result colored by the expansion rate.}
%\label{fig:i1 exp rate}
%\end{figure}
%
%\begin{figure}[htbp!]
%\centering
%\includegraphics[width=0.85\textwidth]{imbalanced_5_exp_rate.pdf}
%\caption{Imbalanced, 5-year dynamic method result colored by the expansion rate.}
%\label{fig:i5 exp rate}
%\end{figure}
%
%\begin{figure}[htbp!]
%\centering
%\includegraphics[width=0.85\textwidth]{imbalanced_10_exp_rate.pdf}
%\caption{Imbalanced, 10-year dynamic method result colored by the expansion rate.}
%\label{fig:i10 exp rate}
%\end{figure}
%
%In the figures we can see that smaller expansion rates, in general, lead to higher accuracy results. For all cases, the higher expansion-rate simulations tend to be clustered towards the right-side of the data cloud, whereas the lower expansion-rate simulations tend to be clustered towards the left-side of the data cloud.
%
%From this analysis it appears that the lower expansion-rate simulations give higher accuracy results. Also, based on the histogram analysis of the Pareto simulations it appears that most of the best simulations have starting and ending number of bins per level of $1 < N_{b,i} \leq 10$. To investigate this, the dynamic data sets were again plotted, except this time for only the 1.25, 1.50, $1.62\ldots$ (golden ratio), and 1.75 expansion rates, with the data colored by the number of bins in the starting level. The balanced load results for 1-, 5-, and 10-year simulations are shown in Figures \ref{fig:b1 start num}, \ref{fig:b5 start num}, and \ref{fig:b10 start num}, respectively. The imbalanced load results for 1-, 5-, and 10-year simulations are shown in Figures \ref{fig:i1 start num}, \ref{fig:i5 start num}, and \ref{fig:i10 start num}, respectively.
%
%\begin{figure}[htbp!]
%\centering
%\includegraphics[width=0.85\textwidth]{balanced_1_125-to-175_exp_rate_start_width.pdf}
%\caption{Balanced, 1-year dynamic method with selected expansion rates and results colored by the number of bins in the first level.}
%\label{fig:b1 start num}
%\end{figure}
%
%\begin{figure}[htbp!]
%\centering
%\includegraphics[width=0.85\textwidth]{balanced_5_125-to-175_exp_rate_start_width.pdf}
%\caption{Balanced, 5-year dynamic method with selected expansion rates and results colored by the number of bins in the first level.}
%\label{fig:b5 start num}
%\end{figure}
%
%\begin{figure}[htbp!]
%\centering
%\includegraphics[width=0.85\textwidth]{balanced_10_125-to-175_exp_rate_start_width.pdf}
%\caption{Balanced, 10-year dynamic method with selected expansion rates and results colored by the number of bins in the first level.}
%\label{fig:b10 start num}
%\end{figure}
%
%\begin{figure}[htbp!]
%\centering
%\includegraphics[width=0.85\textwidth]{imbalanced_1_125-to-175_exp_rate_start_width.pdf}
%\caption{Imbalanced, 1-year dynamic method with selected expansion rates and results colored by the number of bins in the first level.}
%\label{fig:i1 start num}
%\end{figure}
%
%\begin{figure}[htbp!]
%\centering
%\includegraphics[width=0.85\textwidth]{imbalanced_5_125-to-175_exp_rate_start_width.pdf}
%\caption{Imbalanced, 5-year dynamic method with selected expansion rates and results colored by the number of bins in the first level.}
%\label{fig:i5 start num}
%\end{figure}
%
%\begin{figure}[htbp!]
%\centering
%\includegraphics[width=0.85\textwidth]{imbalanced_10_125-to-175_exp_rate_start_width.pdf}
%\caption{Imbalanced, 10-year dynamic method with selected expansion rates and results colored by the number of bins in the first level.}
%\label{fig:i10 start num}
%\end{figure}
%
%Based on the results shown in Figures \ref{fig:b1 start num}-\ref{fig:i10 start num}, it appears that in all cases simulations with a larger number of level-1 bins have higher accuracy, which was expected. As was stated previously, loads which occurred most recently in the GHE's simulation history are expected to have the largest effect on the GHE short-term temperature response. This is the reason why all aggregation methods prioritized keeping the loads most previous to the current time-step unaggregated.
%
%To investigate further, we can further reduce the data by plotting only data which have a uniform number of bins at each level. The result of this for the balanced load simulations is shown in Figures \ref{fig:b1 uniform num}, \ref{fig:b5 uniform num}, and \ref{fig:b10 uniform num}, respectively for the 1-, 5-, and 10-year simulation cases. Figures \ref{fig:i1 uniform num}, \ref{fig:i5 uniform num}, and \ref{fig:i10 uniform num} indicate the same simulation cases except for the imbalanced load cases.
%
%\begin{figure}[htbp!]
%\centering
%\includegraphics[width=0.85\textwidth]{balanced_1_125-to-175_exp_rate_uniform_start_width_end_width.pdf}
%\caption{Balanced, 1-year dynamic method with selected expansion rates and results colored by the number of bins in each level.}
%\label{fig:b1 uniform num}
%\end{figure}
%
%\begin{figure}[htbp!]
%\centering
%\includegraphics[width=0.85\textwidth]{balanced_5_125-to-175_exp_rate_uniform_start_width_end_width.pdf}
%\caption{Balanced, 5-year dynamic method with selected expansion rates and results colored by the number of bins in each level.}
%\label{fig:b5 uniform num}
%\end{figure}
%
%\begin{figure}[htbp!]
%\centering
%\includegraphics[width=0.85\textwidth]{balanced_10_125-to-175_exp_rate_uniform_start_width_end_width.pdf}
%\caption{Balanced, 10-year dynamic method with selected expansion rates and results colored by the number of bins in each level.}
%\label{fig:b10 uniform num}
%\end{figure}
%
%\begin{figure}[htbp!]
%\centering
%\includegraphics[width=0.85\textwidth]{imbalanced_1_125-to-175_exp_rate_uniform_start_width_end_width.pdf}
%\caption{Imbalanced, 1-year dynamic method with selected expansion rates and results colored by the number of bins in each level.}
%\label{fig:i1 uniform num}
%\end{figure}
%
%\begin{figure}[htbp!]
%\centering
%\includegraphics[width=0.85\textwidth]{imbalanced_5_125-to-175_exp_rate_uniform_start_width_end_width.pdf}
%\caption{Imbalanced, 5-year dynamic method with selected expansion rates and results colored by the number of bins in each level.}
%\label{fig:i5 uniform num}
%\end{figure}
%
%\begin{figure}[htbp!]
%\centering
%\includegraphics[width=0.85\textwidth]{imbalanced_10_125-to-175_exp_rate_uniform_start_width_end_width.pdf}
%\caption{Imbalanced, 10-year dynamic method with selected expansion rates and results colored by the number of bins in each level.}
%\label{fig:i10 uniform num}
%\end{figure}
%
%In the figures, we again see that the simulations with a higher number of bins per level all have higher accuracy simulation results. One point that we should note is that there is not a lot of spread in the data regarding runtime fraction. Most of the dynamic results perform similarly with respect to runtime savings. However, in an effort to investigate which (if any) parameters yield better runtime fraction performance, the following plots are presented.
%
%Figure \ref{fig:b1 end num}, \ref{fig:b5 end num}, and \ref{fig:b10 end num} show the 1-, 5-, and 10-year balanced load simulation data with expansion rates of 1.25, 1.50, $1.62\ldots$ (golden ratio), and 1.75 and the number of level-1 bin held constant at 10. Similarly, the same data is plotted in Figures \ref{fig:i1 end num}, \ref{fig:i5 end num}, and \ref{fig:i10 end num} for the 1-, 5-, and 10-year imbalanced load simulation cases, respectively. The data are colored by the number of bins in the final aggregation level.
%
%\begin{figure}[htbp!]
%\centering
%\includegraphics[width=0.85\textwidth]{balanced_1_125-to-175_exp_rate_end_width.pdf}
%\caption{Balanced, 1-year dynamic method with selected expansion rates with 10, level-1 bins and the results colored by the number of bins in the final level.}
%\label{fig:b1 end num}
%\end{figure}
%
%\begin{figure}[htbp!]
%\centering
%\includegraphics[width=0.85\textwidth]{balanced_5_125-to-175_exp_rate_end_width.pdf}
%\caption{Balanced, 5-year dynamic method with selected expansion rates with 10, level-1 bins and the results colored by the number of bins in the final level.}
%\label{fig:b5 end num}
%\end{figure}
%
%\begin{figure}[htbp!]
%\centering
%\includegraphics[width=0.85\textwidth]{balanced_10_125-to-175_exp_rate_end_width.pdf}
%\caption{Balanced, 10-year dynamic method with selected expansion rates with 10, level-1 bins and the results colored by the number of bins in the final level.}
%\label{fig:b10 end num}
%\end{figure}
%
%\begin{figure}[htbp!]
%\centering
%\includegraphics[width=0.85\textwidth]{imbalanced_1_125-to-175_exp_rate_end_width.pdf}
%\caption{Imbalanced, 1-year dynamic method with selected expansion rates with 10, level-1 bins and the results colored by the number of bins in the final level.}
%\label{fig:i1 end num}
%\end{figure}
%
%\begin{figure}[htbp!]
%\centering
%\includegraphics[width=0.85\textwidth]{imbalanced_5_125-to-175_exp_rate_end_width.pdf}
%\caption{Imbalanced, 5-year dynamic method with selected expansion rates with 10, level-1 bins and the results colored by the number of bins in the final level.}
%\label{fig:i5 end num}
%\end{figure}
%
%\begin{figure}[htbp!]
%\centering
%\includegraphics[width=0.85\textwidth]{imbalanced_10_125-to-175_exp_rate_end_width.pdf}
%\caption{Imbalanced, 10-year dynamic method with selected expansion rates with 10, level-1 bins and the results colored by the number of bins in the final level.}
%\label{fig:i10 end num}
%\end{figure}
%
%Figure \ref{fig:b1 end num} shows some slight vertical scattering effects with respect to the number of final-level bins. This may indicate that fewer final-level bins will result in approximately consistent accuracy, but better runtime performance. This will result in fewer aggregation computations which is expected to result in better runtime performance. This vertical scattering of data with respect to the number of final-level bins is not replicated in Figures \ref{fig:b5 end num}--\ref{fig:i10 end num}, though. As a result, it would appear that whether or not the number of bins reduces as aggregation levels advance is not necessarily a strong predictor of improved runtime performance.
%
%\subsection{Simulation-time Related Trends}
%
%Figures \ref{fig:b rmse timing} and \ref{fig:i rmse timing} show the root-mean squared error for the mean fluid temperature vs.~simulation years 1--12 for the balanced and imbalanced cases, respectively. The simulations were performed for the expansion rates of 1.25, 1.50, $1.62\ldots$ (golden ratio), and 1.75 with a uniform number of bins per level which were set to 1, 5, and 10.
%
%\begin{figure}[htbp!]
%\centering
%\includegraphics[width=0.90\textwidth]{timing_rmse_balanced.pdf}
%\caption{Balanced RMSE MFT vs.~number of simulation years for selected expansion rates and number of bins in each level.}
%\label{fig:b rmse timing}
%\end{figure}
%
%\begin{figure}[htbp!]
%\centering
%\includegraphics[width=0.90\textwidth]{timing_rmse_imbalanced.pdf}
%\caption{Imbalanced RMSE MFT vs.~number of simulation years for selected expansion rates and number of bins in each level.}
%\label{fig:i rmse timing}
%\end{figure}
%
%In these figures, we can see that in all cases smaller expansion rates yield better accuracy. Additionally, we can see that having a larger number of bins also yields better accuracy, though the effect decreases significantly when going from $N_b = 5$ to $N_b = 10$. All methods are approximately flat with respect to the RMSE MFT.
%
%Figures \ref{fig:b runtime timing} and \ref{fig:i runtime timing} show the same simulations plotted with runtime improvement plotted vs.~the number of simulation years for the balanced and imbalanced load profiles, respectively. Again, the simulations were performed for the expansion rates of 1.25, 1.50, $1.62\ldots$ (golden ratio), and 1.75 with a uniform number of bins per level which were set to 1, 5, and 10.
%
%\begin{figure}[htbp!]
%\centering
%\includegraphics[width=0.90\textwidth]{timing_runtime_balanced.pdf}
%\caption{Balanced runtime improvement vs.~number of simulation years for selected expansion rates and number of bins in each level.}
%\label{fig:b runtime timing}
%\end{figure}
%
%\begin{figure}[htbp!]
%\centering
%\includegraphics[width=0.90\textwidth]{timing_runtime_imbalanced.pdf}
%\caption{Imbalanced runtime improvement vs.~number of simulation years for selected expansion rates and number of bins in each level.}
%\label{fig:i runtime timing}
%\end{figure}
%
%From the figures, we can see that runtime improvement for all cases is very similar. We can also see that the runtime improvement scales nearly linearly. However, after the 9-year simulations, the slope of the runtime performance decreases. Additionally, the data 8-year simulation data indicates some scatter in runtime improvement. Currently, the cause of both these issues is unknown.
%
%\subsection{Effects of Soil Properties on Load Aggregation}
%
%Figures \ref{fig:soil diff} and \ref{fig:soil cond} show the effects of the soil thermal diffusivity and thermal conductivity on the aggregation results. For these simulations, the dynamic aggregation method was used to perform hourly, 1-year simulations. The number of aggregation bins at each level was fixed at 1, and an expansion rate of 1.25 was used. The soil density, specific heat, and thermal conductivity were all varied $\pm20\%$ at an interval of 5\% from the base value established for the previous simulations. These simulations were all compared against hourly, 1-year simulations with the same variations in soil properties, but with no load aggregation applied.
%
%\begin{figure}[htbp!]
%\centering
%\includegraphics[width=0.85\textwidth]{soil_diffusivity.pdf}
%\caption{Effects of soil thermal diffusivity on aggregation method.}
%\label{fig:soil diff}
%\end{figure}
%
%\begin{figure}[htbp!]
%\centering
%\includegraphics[width=0.85\textwidth]{soil_conductivity.pdf}
%\caption{Effects of soil conductivity on aggregation method.}
%\label{fig:soil cond}
%\end{figure}
%
%Figure \ref{fig:soil diff} shows that the effects of soil thermal diffusivity on this particular aggregation case affect the accuracy of the method to some degree. Higher thermal diffusivity values generally tend towards the left-side of the data cloud, whereas lower diffusivity values generally tend towards the right-side. The effects are more pronounced in Figure \ref{fig:soil cond}, which shows the effects of the soil thermal conductivity on the aggregation method. Here, we can see that higher soil thermal conductivity values result in better accuracy, whereas lower values result in decreases in accuracy. This is expected behavior since lower soil conductivity values will result in larger temperature responses of the GHE. These larger temperature responses, when comparing t for the aggregated and non-aggregated cases, will result in larger temperature errors. When aggregated, these errors result in an overall expected decrease accuracy. Despite this, however, the effects are see are on the order  of 0.02-\SI{0.03}{\celsius}. Soil conductivity values are rarely known with high enough accuracy for this effect to be considered. Therefore, the effects of soil properties on the aggregation method are quite small, and potentially negligible.
%
%\subsection{Short Time-step Effects on Load Aggregation}
%
%Upon testing the aggregation method with short time-step simulations, a few curious results occurred which required investigation. The primary issue can be observed in Figure \ref{fig:sts 5000 before} and is occurring at \SI{3600}{\second}, which is the sudden temperature rise in the simulation data. In this and the following plots, the simulations are compared against the multi flow-rate thermal response test (MFRTRT) data which were collected by this this author and are described in detail in \cite{BEIER201855}.
%
%\begin{figure}[htbp!]
%\centering
%\includegraphics[width=0.85\textwidth]{STS_5000_Before.pdf}
%\caption{MFRTRT short time-step errors with hourly dynamic method for 1.4 simulation hours.}
%\label{fig:sts 5000 before}
%\end{figure}
%
%For this simulation, the dynamic method is being used for load aggregation for times greater than \SI{1}{\hour}. However, the simulation was run at a \SI{2}{\minute} time-step interval. As a result, short time-step (STS) load data for the \SI{2}{\minute} intervals were not being aggregated. Once these data began to be aggregated, the longer time-step aggregation intervals which will naturally compute a larger g-function value cause the simulation temperature rise calculations to experience a sudden increase. As of now, there were no observed errors in either the response factor calculations or the aggregation method, and the behavior is currently understood to be an artifact of how the methods interact with one another.
%
%In an effort to correct the behavior, the aggregation method was changed from beginning at the 1-hour level to beginning at the STS level. So, instead of handling about 30, unaggregated \SI{2}{\minute} duration bins with the beginning aggregation at the 1-hour level, the method began aggregating at the \SI{2}{\minute} level and expanding bin sizes per the dynamic method expansion rate. The results of this change are shown in Figure \ref{fig:sts 5000 after}.
%
%\begin{figure}[htbp!]
%\centering
%\includegraphics[width=0.85\textwidth]{STS_5000_After.pdf}
%\caption{MFRTRT short time-step errors corrected for dynamic method for 1.4 simulation hours.}
%\label{fig:sts 5000 after}
%\end{figure}
%
%This is a minor change in the method, but is important for accurate simulation results near the beginning of the simulation period. On the flip side, though, this is only important for accurate results near the beginning of the simulation. So, for a multi-year simulation, the effects of this change are not expected to be significant. Figures \ref{fig:sts 180000 before} and \ref{fig:sts 180000 after} show the ``before" and ``after" change effects for the first 50 simulation hours for the MFRTRT data. In fact, the ``before" results in Figure \ref{fig:sts 180000 before} show a better agreement with the experimental data over this period than the data shown in Figure \ref{fig:sts 180000 after}. It is currently unknown whether a larger systematic error is occurring which would be causing this.
%
%\begin{figure}[htbp!]
%\centering
%\includegraphics[width=0.85\textwidth]{STS_180000_Before.pdf}
%\caption{MFRTRT short time-step errors with hourly dynamic method for 50 simulation hours.}
%\label{fig:sts 180000 before}
%\end{figure}
%
%\begin{figure}[htbp!]
%\centering
%\includegraphics[width=0.85\textwidth]{STS_180000_After.pdf}
%\caption{MFRTRT short time-step errors corrected for dynamic method for 50 simulation hours.}
%\label{fig:sts 180000 after}
%\end{figure}
%
%\section{Conclusions and Recommendations}
%
%This study has described a large parametric study which was performed to investigate different load aggregation methods for response factor type GHE models. The simulation performed showed that the so termed ``dynamic" method developed by \cite{ClaessonJaved2012} fared best overall. However, some aggregation parameter adjustments are warranted to achieve the highest levels of accuracy and simulation runtime performance.
%
%In summary, the following are expected to be true regarding load aggregation methods.
%
%\begin{enumerate}
%    \item The dynamic method performs best overall when compare to the the static method.
%    \item Smaller expansion rates tend to yield higher accuracy results. Expansion rates $<2$ will yield acceptable accuracies. However, the expansion rate of 1.25---which is the lowest expansion rate tested in this study---consistently yielded accuracy levels $<$ \SI{0.1}{\celsius}.
%    \item The number of aggregation bins per level has a larger effect on simulation accuracy than on simulation runtime. A value of $N_b=5$ is recommended. It is not necessary to vary the number of aggregation bins per level to achieve high accuracy, efficient simulations.
%    \item In general, the dynamic method yields competitive simulation times regardless of the specific aggregation parameters. However, some simulation time efficiency may be gained by tapering the number of aggregation bins from, say, 5 at the first level to 1 at the final level.
%    \item Soil properties may play some effect in the accuracy of the load aggregation method. However, due to the fact that the soil properties are rarely know with a high degree of accuracy, any variations due to soil property variations are likely negligible.
%    \item Implementing the dynamic method for short time-step simulation may need additional consideration to accurately consider the effects of STS calculations.
%\end{enumerate}
%
%\section{Acknowledgments}
%The work performed was partially supported under the National Renewable Energy Laboratory Task Order Agreement No.~KAGN-4-42503-00. The computing for the project was performed at the OSU High Performance Computing Center at Oklahoma State University supported in part through the National Science Foundation grant OAC-1126330.

\bibliographystyle{model2-names}\biboptions{authoryear}
\bibliography{../../../Writing/References/References}

\end{document}
